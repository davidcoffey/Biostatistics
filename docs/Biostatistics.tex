\documentclass[]{book}
\usepackage{lmodern}
\usepackage{amssymb,amsmath}
\usepackage{ifxetex,ifluatex}
\usepackage{fixltx2e} % provides \textsubscript
\ifnum 0\ifxetex 1\fi\ifluatex 1\fi=0 % if pdftex
  \usepackage[T1]{fontenc}
  \usepackage[utf8]{inputenc}
\else % if luatex or xelatex
  \ifxetex
    \usepackage{mathspec}
  \else
    \usepackage{fontspec}
  \fi
  \defaultfontfeatures{Ligatures=TeX,Scale=MatchLowercase}
\fi
% use upquote if available, for straight quotes in verbatim environments
\IfFileExists{upquote.sty}{\usepackage{upquote}}{}
% use microtype if available
\IfFileExists{microtype.sty}{%
\usepackage{microtype}
\UseMicrotypeSet[protrusion]{basicmath} % disable protrusion for tt fonts
}{}
\usepackage[margin=1in]{geometry}
\usepackage{hyperref}
\hypersetup{unicode=true,
            pdftitle={Biostatistics},
            pdfauthor={David Coffey},
            pdfborder={0 0 0},
            breaklinks=true}
\urlstyle{same}  % don't use monospace font for urls
\usepackage{natbib}
\bibliographystyle{apalike}
\usepackage{color}
\usepackage{fancyvrb}
\newcommand{\VerbBar}{|}
\newcommand{\VERB}{\Verb[commandchars=\\\{\}]}
\DefineVerbatimEnvironment{Highlighting}{Verbatim}{commandchars=\\\{\}}
% Add ',fontsize=\small' for more characters per line
\usepackage{framed}
\definecolor{shadecolor}{RGB}{248,248,248}
\newenvironment{Shaded}{\begin{snugshade}}{\end{snugshade}}
\newcommand{\KeywordTok}[1]{\textcolor[rgb]{0.13,0.29,0.53}{\textbf{#1}}}
\newcommand{\DataTypeTok}[1]{\textcolor[rgb]{0.13,0.29,0.53}{#1}}
\newcommand{\DecValTok}[1]{\textcolor[rgb]{0.00,0.00,0.81}{#1}}
\newcommand{\BaseNTok}[1]{\textcolor[rgb]{0.00,0.00,0.81}{#1}}
\newcommand{\FloatTok}[1]{\textcolor[rgb]{0.00,0.00,0.81}{#1}}
\newcommand{\ConstantTok}[1]{\textcolor[rgb]{0.00,0.00,0.00}{#1}}
\newcommand{\CharTok}[1]{\textcolor[rgb]{0.31,0.60,0.02}{#1}}
\newcommand{\SpecialCharTok}[1]{\textcolor[rgb]{0.00,0.00,0.00}{#1}}
\newcommand{\StringTok}[1]{\textcolor[rgb]{0.31,0.60,0.02}{#1}}
\newcommand{\VerbatimStringTok}[1]{\textcolor[rgb]{0.31,0.60,0.02}{#1}}
\newcommand{\SpecialStringTok}[1]{\textcolor[rgb]{0.31,0.60,0.02}{#1}}
\newcommand{\ImportTok}[1]{#1}
\newcommand{\CommentTok}[1]{\textcolor[rgb]{0.56,0.35,0.01}{\textit{#1}}}
\newcommand{\DocumentationTok}[1]{\textcolor[rgb]{0.56,0.35,0.01}{\textbf{\textit{#1}}}}
\newcommand{\AnnotationTok}[1]{\textcolor[rgb]{0.56,0.35,0.01}{\textbf{\textit{#1}}}}
\newcommand{\CommentVarTok}[1]{\textcolor[rgb]{0.56,0.35,0.01}{\textbf{\textit{#1}}}}
\newcommand{\OtherTok}[1]{\textcolor[rgb]{0.56,0.35,0.01}{#1}}
\newcommand{\FunctionTok}[1]{\textcolor[rgb]{0.00,0.00,0.00}{#1}}
\newcommand{\VariableTok}[1]{\textcolor[rgb]{0.00,0.00,0.00}{#1}}
\newcommand{\ControlFlowTok}[1]{\textcolor[rgb]{0.13,0.29,0.53}{\textbf{#1}}}
\newcommand{\OperatorTok}[1]{\textcolor[rgb]{0.81,0.36,0.00}{\textbf{#1}}}
\newcommand{\BuiltInTok}[1]{#1}
\newcommand{\ExtensionTok}[1]{#1}
\newcommand{\PreprocessorTok}[1]{\textcolor[rgb]{0.56,0.35,0.01}{\textit{#1}}}
\newcommand{\AttributeTok}[1]{\textcolor[rgb]{0.77,0.63,0.00}{#1}}
\newcommand{\RegionMarkerTok}[1]{#1}
\newcommand{\InformationTok}[1]{\textcolor[rgb]{0.56,0.35,0.01}{\textbf{\textit{#1}}}}
\newcommand{\WarningTok}[1]{\textcolor[rgb]{0.56,0.35,0.01}{\textbf{\textit{#1}}}}
\newcommand{\AlertTok}[1]{\textcolor[rgb]{0.94,0.16,0.16}{#1}}
\newcommand{\ErrorTok}[1]{\textcolor[rgb]{0.64,0.00,0.00}{\textbf{#1}}}
\newcommand{\NormalTok}[1]{#1}
\usepackage{longtable,booktabs}
\usepackage{graphicx,grffile}
\makeatletter
\def\maxwidth{\ifdim\Gin@nat@width>\linewidth\linewidth\else\Gin@nat@width\fi}
\def\maxheight{\ifdim\Gin@nat@height>\textheight\textheight\else\Gin@nat@height\fi}
\makeatother
% Scale images if necessary, so that they will not overflow the page
% margins by default, and it is still possible to overwrite the defaults
% using explicit options in \includegraphics[width, height, ...]{}
\setkeys{Gin}{width=\maxwidth,height=\maxheight,keepaspectratio}
\IfFileExists{parskip.sty}{%
\usepackage{parskip}
}{% else
\setlength{\parindent}{0pt}
\setlength{\parskip}{6pt plus 2pt minus 1pt}
}
\setlength{\emergencystretch}{3em}  % prevent overfull lines
\providecommand{\tightlist}{%
  \setlength{\itemsep}{0pt}\setlength{\parskip}{0pt}}
\setcounter{secnumdepth}{5}
% Redefines (sub)paragraphs to behave more like sections
\ifx\paragraph\undefined\else
\let\oldparagraph\paragraph
\renewcommand{\paragraph}[1]{\oldparagraph{#1}\mbox{}}
\fi
\ifx\subparagraph\undefined\else
\let\oldsubparagraph\subparagraph
\renewcommand{\subparagraph}[1]{\oldsubparagraph{#1}\mbox{}}
\fi

%%% Use protect on footnotes to avoid problems with footnotes in titles
\let\rmarkdownfootnote\footnote%
\def\footnote{\protect\rmarkdownfootnote}

%%% Change title format to be more compact
\usepackage{titling}

% Create subtitle command for use in maketitle
\newcommand{\subtitle}[1]{
  \posttitle{
    \begin{center}\large#1\end{center}
    }
}

\setlength{\droptitle}{-2em}

  \title{Biostatistics}
    \pretitle{\vspace{\droptitle}\centering\huge}
  \posttitle{\par}
    \author{David Coffey}
    \preauthor{\centering\large\emph}
  \postauthor{\par}
      \predate{\centering\large\emph}
  \postdate{\par}
    \date{2018-06-29}

\usepackage{booktabs}
\usepackage{amsthm}
\makeatletter
\def\thm@space@setup{%
  \thm@preskip=8pt plus 2pt minus 4pt
  \thm@postskip=\thm@preskip
}
\makeatother

\usepackage{amsthm}
\newtheorem{theorem}{Theorem}[chapter]
\newtheorem{lemma}{Lemma}[chapter]
\theoremstyle{definition}
\newtheorem{definition}{Definition}[chapter]
\newtheorem{corollary}{Corollary}[chapter]
\newtheorem{proposition}{Proposition}[chapter]
\theoremstyle{definition}
\newtheorem{example}{Example}[chapter]
\theoremstyle{definition}
\newtheorem{exercise}{Exercise}[chapter]
\theoremstyle{remark}
\newtheorem*{remark}{Remark}
\newtheorem*{solution}{Solution}
\begin{document}
\maketitle

{
\setcounter{tocdepth}{1}
\tableofcontents
}
\chapter{General overview}\label{general-overview}

\section{Introduction}\label{introduction}

This books provides a consise overview of biostatistics and its
applications using the R programming language. The textbook
\emph{Fundamentals of Biostatitics} \citep{Rosner2016} was used
extensivity in the preparation of this book.

\section{Example dataset}\label{example-dataset}

Examples of R functions are performed on a dataset of patients with
newly diagnosed multiple myeloma. This dataset contains a variety of
categorical and continuous variables (Table 1).

ID

Sex

Race

Age

Stage

SurvivalMonths

Status

DiagnosisYear

Treatment

TreatmentDurationMonths

BonyLesions

PlasmaCells

1q+

del13q

del17p

del1p

t(11;14)

t(14;16)

t(4:14)

t(6;14)

Albumin

B2M

Calcium

Creatinine

LightChainRatio

Hematocrit

LDH

MProtein

1

Male

Black

65

II

20.60

Unknown

2016

VRD

78

1

80.000

Normal

Normal

Normal

Normal

Normal

Normal

Abnormal

Normal

3.1

4.7

9.6

1.02

135.42

28

208

4.90

2

Female

White

44

I

16.23

Dead

2015

VRD

79

\textgreater{}3

20.000

Normal

Normal

Abnormal

Abnormal

Normal

Normal

Normal

Normal

4.8

1.5

9.6

0.72

3700.00

34

183

0.00

3

Male

Black

55

II

22.63

Alive

2016

VRD

173

\textgreater{}3

30.000

Normal

Normal

Normal

Normal

Normal

Normal

Normal

Normal

3.4

4.3

9.1

1.28

5.82

33

127

4.20

4

Female

White

64

I

22.63

Alive

2016

VRD

184

0

54.000

Normal

Normal

Normal

Normal

Normal

Normal

Normal

Normal

4.2

2.6

10.1

0.71

4.77

31

190

1.70

5

Female

White

62

III

21.30

Alive

2016

VRD

93

1

0.028

Normal

Normal

Normal

Abnormal

Normal

Normal

Normal

Normal

4.6

6.7

10.0

0.90

113.38

32

243

0.30

6

Male

White

64

III

17.57

Alive

2016

CyBorD

21

0

17.800

Normal

Abnormal

Normal

Normal

Normal

Normal

Normal

Normal

2.1

17.0

13.0

3.84

2105.97

20

100

7.30

7

Female

White

60

II

35.43

Unknown

2012

Not specified

528

0

0.000

Normal

Normal

Normal

Normal

Abnormal

Normal

Normal

Normal

4.3

4.4

11.4

1.01

15575.00

35

188

0.00

8

Male

White

58

II

27.83

Alive

2016

VRD

76

\textgreater{}3

5.000

Normal

Abnormal

Normal

Normal

Abnormal

Normal

Normal

Normal

4.1

2.3

9.3

0.93

44.80

44

97

1.10

9

Male

White

69

I

37.70

Alive

2015

Not specified

27

\textgreater{}3

9.600

Normal

Abnormal

Normal

Normal

Normal

Normal

Normal

Normal

4.5

2.3

9.2

0.91

82.20

40

205

0.00

10

Male

White

51

III

31.83

Alive

2015

CyBorD

192

0

43.000

Normal

Normal

Normal

Normal

Abnormal

Normal

Normal

Normal

4.3

8.6

16.0

3.90

148.15

37

253

0.70

11

Female

White

33

II

58.10

Unknown

2011

VRD

73

\textgreater{}3

0.000

Normal

Normal

Normal

Normal

Normal

Normal

Normal

Normal

2.6

2.4

8.6

0.79

1.94

31

180

6.50

12

Male

White

57

I

22.63

Unknown

2015

Not specified

24

\textgreater{}3

65.000

Normal

Abnormal

Normal

Normal

Normal

Normal

Normal

Normal

4.2

2.9

9.7

0.97

464.86

38

122

1.60

13

Female

Black

72

I

10.23

Unknown

2012

Not specified

55

1

21.000

Normal

Normal

Normal

Normal

Normal

Abnormal

Normal

Normal

4.0

5.5

9.9

1.10

1030.00

31

1551

0.00

14

Female

White

64

II

43.57

Alive

2014

Not specified

1979

\textgreater{}3

68.000

Normal

Normal

Normal

Normal

Normal

Normal

Normal

Normal

2.9

3.6

11.4

0.57

1860.00

32

89

2.90

15

Male

Not reported

63

III

52.77

Alive

2014

VRD

1016

0

38.000

Normal

Normal

Normal

Normal

Normal

Normal

Abnormal

Normal

2.6

5.7

9.2

1.07

75.82

30

104

4.30

16

Male

Asian

52

III

27.23

Alive

2016

VRD

162

0

55.000

Normal

Abnormal

Abnormal

Normal

Normal

Normal

Abnormal

Normal

3.7

3.9

9.2

1.08

741.18

22

172

0.25

17

Male

White

76

II

22.37

Dead

2014

Not specified

88

\textgreater{}3

25.000

Normal

Normal

Normal

Normal

Normal

Normal

Normal

None

3.0

5.3

9.9

1.27

21.05

34

130

3.90

18

Female

White

64

III

33.87

Unknown

2014

VRD

212

\textgreater{}3

0.000

Abnormal

Normal

Normal

Abnormal

Normal

Normal

Normal

Normal

4.4

1.9

10.0

0.83

1.93

43

241

0.00

19

Male

White

53

I

27.73

Unknown

2014

VRD

78

1

0.000

Normal

Abnormal

Normal

Normal

Normal

Normal

Normal

Normal

3.7

1.7

9.6

0.86

3.44

40

139

3.10

20

Male

White

64

II

84.70

Dead

2009

CyBorD

60

0

50.000

Normal

Normal

Normal

Normal

Normal

Normal

Normal

Normal

3.8

2.1

10.1

1.10

97.46

35

192

1.10

\chapter{Descriptive statistics}\label{descriptive-statistics}

\section{Arithmetic mean}\label{arithmetic-mean}

The arithmetic mean (\(\bar{x}\)) is a measure of central location. It
is calculated from the sum of all the observations (\({n}\)) divided by
the number of observations:

\[\bar{x}=\frac{\sum_{i=1}^nx_{i}}{n}\]

The notation \(\sum_{i=1}^nx_{i}\) means the sum of all \({x}_i\)
observations \(({x}_1+{x}_2+{x}_n)\). One limitation to the arithmetic
mean is that it is overly sensitive to extreme values.

\begin{Shaded}
\begin{Highlighting}[]
\CommentTok{# Import dataset}
\KeywordTok{load}\NormalTok{(}\StringTok{"docs/Example-data.Rda"}\NormalTok{)}

\CommentTok{# Calcuate arithmetic mean}
\KeywordTok{mean}\NormalTok{(data}\OperatorTok{$}\NormalTok{Age)}
\end{Highlighting}
\end{Shaded}

\begin{verbatim}
[1] 59.5
\end{verbatim}

\section{Median}\label{median}

If all observations are ordered from smallest to largest, the median is
the middle number. More precisely, if \({n}\) is odd, \(\frac{n+1}{2}\),
or if \({n}\) is even, the average of \(\frac{n}{2}\) and
\(\frac{n}{2}+1\).

The rationale for using to the median is to ensure an equal number of
observations on both sides of the sample median. The main weakness of
the sample median is that it is less sensitive to the actual numeric
values of the data points. If the sample distribution is symmetric, the
arithmetic mean is approximately the same as the median. For positively
skewed distributions, the arithmetic mean tends to be larger than the
median; for negatively skewed distributions, the arithmetic means tends
to be smaller than the median.

\begin{Shaded}
\begin{Highlighting}[]
\CommentTok{# Calcuate arithmetic mean}
\KeywordTok{median}\NormalTok{(data}\OperatorTok{$}\NormalTok{Age)}
\end{Highlighting}
\end{Shaded}

\begin{verbatim}
[1] 62.5
\end{verbatim}

\section{Mode}\label{mode}

The mode is the most frequently occurring value among all of the
observations in a sample. Some distributions have more than one mode. A
distribution with one mode is called unimodal; two modes, bimodal; three
modes, trimodal.

\begin{Shaded}
\begin{Highlighting}[]
\CommentTok{# Calcuate mode}
\KeywordTok{library}\NormalTok{(DescTools)}
\KeywordTok{Mode}\NormalTok{(data}\OperatorTok{$}\NormalTok{Age)}
\end{Highlighting}
\end{Shaded}

\begin{verbatim}
[1] 64
\end{verbatim}

\section{Geometric mean}\label{geometric-mean}

The geometric mean (\(\bar{logx}\)) is the central number in a geometric
progression such as exponential growth. The geometric mean is defined as
the \({n}\)th root of the product of \({n}\) numbers:

\[\bar{logx}=\frac{\sum_{i=1}^nlogx_{i}}{n}\] Any base can be used to
compute the logarithms for the geometric mean. It is usually preferable
to work in the original scale by taking the antilogarithm of
\(\bar{logx}\) to form the geometric mean.

\begin{Shaded}
\begin{Highlighting}[]
\CommentTok{# Calcuate geometric arithmetic mean}
\KeywordTok{library}\NormalTok{(DescTools)}
\KeywordTok{Gmean}\NormalTok{(data}\OperatorTok{$}\NormalTok{Age)}
\end{Highlighting}
\end{Shaded}

\begin{verbatim}
[1] 58.61499
\end{verbatim}

\bibliography{Citations.bib}


\end{document}
